%Este trabalho está licenciado sob a Licença Creative Commons Atribuição-CompartilhaIgual 3.0 Não Adaptada. Para ver uma cópia desta licença, visite https://creativecommons.org/licenses/by-sa/3.0/ ou envie uma carta para Creative Commons, PO Box 1866, Mountain View, CA 94042, USA.

\chapter{Limites}\label{cap:limites}\index{limites!de funções}

\section{Noções de topologia}

\begin{defn}\normalfont{(Ponto interior)}
  Diz-se que $x$ é um \emph{ponto interior}\index{ponto interior} de um dado conjunto $C$ quando existe um intervalo $(a, b)$ que contém $x$ e está contido em $C$, i.e. $x\in (a, b)\subset C$. O conjunto de todos os pontos interiores de $C$ é chamado de seu \emph{interior}\index{conjunto!interior}.
\end{defn}

\begin{ex}
  \begin{enumerate}[a)]
  \item Todo elemento de um intervalo aberto $(a, b)$ é ponto interior deste.
  \item O interior de um dado intervalo fechado $[a, b]$ é o intervalo aberto $(a, b)$.
  \end{enumerate}
\end{ex}

\begin{defn}\normalfont{(Conjunto aberto)}
  Diz se que $C$ é \emph{conjunto aberto}\index{conjunto aberto} quando todos seus elementos são pontos interiores.
\end{defn}

\begin{ex}\label{ex:conjunto_aberto}
  Vejamos os seguintes casos:
  \begin{enumerate}[a)]
  \item O intervalo $(a, b) := \{x\in\mathbb{R};~a<x<b\}$ é um conjunto aberto. De fato, dado $x\in (a, b)$ podemos tomar $0 < \epsilon < \min\{x-a,b-x\}$ de forma que $x\in (x-\epsilon, x+\epsilon)\subset (a, b)$.
  \item O intervalo $(a, b]$ não é aberto, pois $b\in (a, b]$ não é ponto interior.
  \item O conjunto vazio $\emptyset$ é um conjunto aberto. Com efeito, se o conjunto $\emptyset$ não é aberto, então existe um elemento $x\in\emptyset$ que não é ponto interior de $\emptyset$, o que é um absurdo pois $\emptyset$ não contém elementos por definição.
  \item O conjunto dos números racionais $\mathbb{Q}$ não é aberto.
  \end{enumerate}
\end{ex}

\begin{defn}\normalfont{(Vizinhança)}
  Uma \emph{vizinhança}\index{vizinhança} de um dado ponto $x$ é qualquer conjunto $V$ que contenha $x$ como ponto interior. Também, a \emph{vizinhança simétrica}\index{vizinhança!simétrica} de um ponto $x\in\mathbb{R}$ é todo intervalo $V_\epsilon(x) := (x-\epsilon, x+\epsilon)$ com $\epsilon>0$. Mais estrito, a \emph{vizinhança perfurada}\index{vizinhança!perfurada} de $x\in\mathbb{R}$ é uma vizinhança de $x$ que não contém $x$. Aproveitamos para fixar a notação:
  \begin{equation*}
    V'_\epsilon(x) := V_\epsilon(x)\setminus \{x\} = \{y\in\mathbb{R};~0<|x-y|<\epsilon\}.
  \end{equation*}
\end{defn}

\begin{ex}
  Podemos reescrever o Exemplo~\ref{ex:conjunto_aberto} da seguinte forma. Um intervalo $(a, b)$ é um conjunto aberto, pois para cada $x\in (a, b)$ podemos escolher $0<\epsilon<\min\{x-a,b-x\}$ tal que $V_\epsilon(x)\subset (a, b)$.
\end{ex}

\begin{defn}\normalfont{(Ponto de acumulação)}
  Um ponto $x$ é chamado de \emph{ponto de acumulação}\index{ponto de acumulação} de um dado conjunto $C$ quando toda vizinhança de $x$ contém infinitos pontos de $C$.
\end{defn}

\begin{ex}
  Vejamos os seguintes casos:
  \begin{enumerate}[a)]
  \item O número $a$ é ponto de acumulação do intervalo $(a, b]$ não degenerado. De fato, dado $\epsilon>0$, temos $(a, a+\epsilon)\subset V_\epsilon(a)$ e $(a, a+\epsilon)\cap (a, b]$ é um conjunto infinito.
  \item Zero é o único ponto de acumulação do conjunto $\{1, 1/2, 1/3, \dotsc, 1/n, \ldots\}$.
  \end{enumerate}
\end{ex}

\begin{defn}\normalfont{(Ponto isolado)}
  Diz que $x$ é \emph{ponto isolado}\index{ponto!isolado} de um dado conjunto $C$ quando $x\in C$ não é ponto de acumulação de $C$. Diz-se que um conjunto é \emph{discreto}\index{conjunto!discreto} quando todos seus elementos são pontos discretos.
\end{defn}

\begin{ex}
  Vejamos os seguintes casos:
  \begin{enumerate}[a)]
  \item O conjunto dos números naturais $\mathbb{N}$ é discreto.
  \item O conjunto dos números racionais $\mathbb{Q}$ não é discreto.
  \item O conjunto $\{1, 1/2, 1/3, \dotsc, 1/n, \ldots\}$ é discreto.
  \end{enumerate}
\end{ex}

\begin{defn}\normalfont{(Ponto aderente)}
  Dizemos que $x$ é \emph{ponto aderente}\index{ponto aderente} de um dado conjunto $C$ quando toda vizinhança de $x$ contém algum ponto de $C$. O conjunto de todos os pontos aderentes de $C$ é chamado de \emph{fecho}\index{fecho} (ou, conjunto de aderência\index{conjunto de!aderência}) de $C$, o qual denotamos por $\overline{C}$.
\end{defn}

\begin{obs}
  Observe que todo ponto de um conjunto é aderente ao mesmo, bem como, todos os seus pontos de acumulação.
\end{obs}

\begin{ex}
  Vejamos os seguintes casos:
  \begin{enumerate}[a)]
  \item O fecho de $(a, b]$ é o intervalo fechado $[a, b]$.
  \item O conjunto dos números reais $\mathbb{R}$ é o fecho do conjunto dos números racionais $\mathbb{Q}$, i.e. $\overline{Q} = \mathbb{R}$.
  \end{enumerate}
\end{ex}

\begin{defn}\normalfont{Conjunto fechado}
  Dizemos que um conjunto $C$ é \emph{fechado}\index{conjunto!fechado} quando é igual ao seu fecho, i.e. $C = \overline{C}$.
\end{defn}

\begin{ex}
  Vejamos os seguintes casos:
  \begin{enumerate}[a)]
  \item O intervalo $[a, b]$ é um conjunto fechado.
  \item O conjunto vazio $\emptyset$ é fechado. Por quê?
  \item O conjunto dos números reais $\mathbb{R}$ é fechado.
  \item O conjunto dos números racionais $\mathbb{Q}$ não é fechado.
  \end{enumerate}
\end{ex}

\begin{defn}\normalfont{(Conjunto denso)}
  Dizemos que um conjunto $A$ é \emph{denso}\index{denso} no conjunto $B$, quando todo ponto aderente de $\overline{A} \subset B$. 
\end{defn}

\begin{ex}
  O conjunto dos números racionais $\mathbb{Q}$ é denso no conjunto dos números reais $\mathbb{R}$.
\end{ex}

\subsection{Exercícios resolvidos}

\construirExeresol

\subsection{Exercícios}

\construirExer

\begin{exer}
  Seja dado um conjunto $C$. Mostre que $x$ é ponto de acumulação de $C$ se, e somente se, toda vizinhança de $x$ contém pelo menos um elemento de $C$ diferente de $x$.
\end{exer}
\begin{sol}
  Basta considerar sucessivas vizinhanças $V_{1/n}(x)$ com $n\in\mathbb{R}$.
\end{sol}

\begin{exer}
  Seja dado um conjunto $C$. Mostre que $x$ é ponto isolado de $C$ se, e somente se, existe uma vizinhança de $x$ tal que $(V(x)\setminus\{x\})\cap C = \emptyset$.
\end{exer}
\begin{sol}
  A implicação segue imediatamente por negação.
\end{sol}


\section{Limites}

\construirExer

\section{Exercícios finais}

\construirExer